The SGPE architecture strictly decouples linguistic integrity from statistical compression through two sequential layers. Layer 1 (\textbf{LinguisTrie}) is a deterministic linguistic barrier that enforces script invariants in linear time. Layer 2 (\textbf{GPE}) performs pair merging exclusively over the resulting stream of well-formed syllables.

\begin{figure}[h]
  \centering
  \includegraphics[scale=0.09]{FIGS/ARCHITECTURE.png}
  \caption{SGPE architecture. Linguistic constraints are enforced \emph{before} any statistical procedure, guaranteeing that no token ever violates Sinhala syllable structure.}
  \label{fig:arch}
\end{figure}

\subsection{Layer 1: LinguisTrie}

LinguisTrie realizes the deterministic finite automaton of Section 3 as a single-pass control-flow state machine. Rather than a table-driven transition function, transitions are encoded directly in conditional branches. This design exploits modern CPU branch prediction and cache locality while guaranteeing strictly monotonic advancement of the input pointer.

The algorithm processes the input in a single left-to-right scan. For each outer iteration it identifies a syllable head (consonant or independent vowel) and enters a greedy lookahead loop that absorbs zero or more conjunct extensions of the form $(\text{H} \text{Z}? \text{C})$ followed by optional terminal modifiers. An optional \texttt{leading\_space} mode prefixes whitespace to the subsequent syllable token, preserving word-boundary context for Layer 2. Consecutive whitespaces and newlines are emitted as structural passthrough tokens, prefixing at most a single space to the syllable.

\begin{definition}[Zero-Breakage Guarantee]
A tokenizer $T$ satisfies the zero-breakage guarantee if, for every input string $w$, each emitted token $t_i \in T(w)$ is either a complete, valid syllable according to the regular language $S$ defined in Section 3 or a single passthrough character from class $O$.
\end{definition}

\begin{theorem}
LinguisTrie satisfies the zero-breakage guarantee for any valid Unicode string.
\end{theorem}

\begin{proof}
By induction on input length. The base case (empty or single-character input) holds trivially. For the inductive step, the consonant branch matches the production $\text{C} (\text{H} \text{Z}? \text{C})^* (\text{P} \mid \text{H})? \text{M}?$, the vowel branch matches $\text{V} \text{M}?$, the orphan branch matches the singleton orphan productions, and all remaining characters fall to the passthrough case. Because the inner conjunct loop exits only when no further valid extension exists and every emitted span is emitted exactly once, no valid syllable is ever split and no invalid token is produced. 
\end{proof}

This guarantee is empirically validated on an exhaustive battery of 1,703 conjunct formations (0 violations) and the full 59.3-million-character evaluation corpus (perfect structural soundness and lossless round-trip for all non-UNK tokens).

\subsection{Layer 2: Grapheme Pair Encoding (GPE)}

Layer 2 applies a constrained BPE procedure to the syllable stream produced by Layer 1. Three modifications distinguish GPE from standard subword algorithms:

\begin{enumerate}
\item \textbf{Syllabic initialization.} The base vocabulary is initialized with atomic syllables and passthrough characters emitted by LinguisTrie. By construction, this ensures that the tokenizer never fractures text into sub-character bytes or meaningless codepoint fragments.
\item \textbf{Boundary-aware scoping.} Merges are performed exclusively within word spans delimited by whitespace or non-Sinhala passthrough tokens. This prevents the formation of cross-word tokens that would otherwise violate morphological boundaries.
\item \textbf{Frequency pruning.} Syllables with a corpus frequency below a threshold $\theta$ are replaced by an \texttt{[UNK]} sentinel prior to the merge procedure. This eliminates the influence of rare or noisy linguistic fragments on statistical merge priority.
\end{enumerate}

Unlike standard BPE and Unigram LM, which operate directly on codepoints (or bytes) and therefore routinely produce conjunct/ZWJ fragmentation, orphan pili/HAL tokens, and cross-word merges, GPE’s constraints rule out these pathologies by construction: every base unit is already linguistically valid, and every merge preserves that validity.

During inference, Layer 1 first segments the input, words are reconstructed via boundary detection, out-of-vocabulary syllables are replaced by \texttt{[UNK]}, and the learned merge rules are applied inside each word using a priority-based single-best-merge loop that respects the order in which merges were learned.

\subsection{Inheritance of Linguistic Integrity}

\begin{corollary}
Excluding special and passthrough tokens, every token in the final SGPE vocabulary is a concatenation of one or more complete syllables produced by LinguisTrie and therefore inherits the zero-breakage guarantee.
\end{corollary}

To illustrate the end-to-end flow, consider the input ``\sn{ශ්‍රී ලංකාව}''. Layer 1 emits the syllable sequence \texttt{["\sn{ශ්‍රී}", "\sn{ ලං}", "\sn{කා}", "\sn{ව}"]}, Layer 2 groups them into two word spans and merges the second span into the single token ``\sn{ ලංකාව}'', yielding the final token sequence \texttt{["\sn{ශ්‍රී}", "\sn{ ලංකාව}"]}. No conjunct, pili, or virama is ever fragmented.

Detailed implementation, pseudocode, and reproducibility artifacts are provided in the companion repository.